\documentclass[12pt,a4paper]{article}
\usepackage[utf8]{inputenc}
\usepackage[T1]{fontenc}
\usepackage[russian]{babel}
\usepackage{amsmath}
\usepackage{amsfonts}
\usepackage{amssymb}
\usepackage{graphicx}
\usepackage[left=2.50cm, right=2.00cm, top=2.00cm, bottom=2.00cm]{geometry}

\usepackage[most]{tcolorbox}
\usepackage{array}
\usepackage{multirow}

\usepackage{longtable}
\usepackage{colortbl}

\newcolumntype{A}{>{\bfseries}p{3.0cm}}
\newcolumntype{B}{>{\it}p{3.5cm}}
\newcolumntype{C}{>{}p{8.5cm}}

\begin{document}

\begin{flushright}
Протокол от 28 июня 2018 г. \textnumero 13 \\
Срок действия программы: 2018-2022 уч.г.
\end{flushright}

\begin{tcolorbox}[colback=gray!10]
\begin{center}
{\bfseries\large
Рабочая программа дисциплины \\ 
"<Пакеты прикладных программ">
}
\end{center}
\end{tcolorbox}

\noindent
Программу составил(и): \\
к.ф.-м.н., доцент \textit{Сидоров В. В.},
к.ф.-м.н., доцент \textit{Иванов М. С.}

\noindent
\begin{tabular}{A B !{\vrule width 2pt\relax} C}
Кафедра & Физика & \multirow{10}{=}{\begin{tabular}{| l | c | c | c | c |}\hline
Семестр & \multicolumn{2}{c|}{4} & \multicolumn{2}{c|}{\multirow{2}{*}{Итого}} \\\cline{1-3}
Количество недель & \multicolumn{2}{c|}{18} & \multicolumn{2}{c|}{ }\\
\hline
Виды занятий & УП & РПД & УП & РПД \\\hline
Лекций & 18 & 18 & 18 & 18 \\\hline
Лабораторные & 18 & 18 & 18 & 18 \\\hline
В том числе инт. & 12 & 12 & 12 & 12 \\\hline
Итого ауд. & 36 & 36 & 36 & 36 \\\hline
Конт. работа & 36 & 36 & 36 & 36 \\\hline
Самост. работа & 18 & 18 & 18 & 18 \\\hline
Часы на контроль & 18 & 18 & 18 & 18 \\\hline
\hline Итого & 72 & 72 & 72 & 72 \\\hline
\end{tabular}
} \\
Учебный план  & 03.03.02-О-БФ-ИСФ-14\_(+3) & \\
Направление & 03.03.02 Физика & \\
Профиль & Информационные системы в физике & \\
Форма обучения & очная & \\
Семестр & 4 & \\
Трудоёмкость & 2 ЗЕТ & \\
Всего часов & 72 & \\
Вид контроля & зачёт & \\
\end{tabular}

\begin{longtable}{|p{2.0cm}|p{13.5cm}|}
\multicolumn{2}{c}{\bfseries \rule{0pt}{0.5cm} Начало таблицы}\\\hline
\endfirsthead
\multicolumn{2}{c}{\bfseries \rule{0pt}{0.5cm} Продолжение таблицы}\\\hline
\endhead
\multicolumn{2}{c}{\bfseries \rule{0pt}{0.5cm} Продолжение таблицы на следующей странице}\\
\endfoot
\multicolumn{2}{c}{\bfseries \rule{0pt}{0.5cm} Конец таблицы}\\
\endlastfoot
\rowcolor[gray]{0.9}
\multicolumn{2}{|l|}{\bfseries \rule{0pt}{0.5cm} 1.  ЦЕЛИ ОСВОЕНИЯ ДИСЦИПЛИНЫ}\\\hline
1.1 & Получение навыков работы в издательской системе \LaTeX. Формирование представления о системах компьютерной математики. Получение практических навыков инженерных расчётов в Maple. \\
\rowcolor[gray]{0.9}
\multicolumn{2}{|l|}{\bfseries \rule{0pt}{0.5cm} 2. МЕСТО ДИСЦИПЛИНЫ В СТРУКТУРЕ ООП}\\\hline
2.1 & Требования к предварительной подготовке обучающегося: \\\hline
2.2 & Базовая подготовка по курсу общей физики. \\\hline
2.3 & Базовая подготовка по высшей математике. \\\hline
2.4 & Навыки работы с персональным компьютером. \\\hline
2.5 & Владение навыками разработки программного обеспечения и опытом работы в инструментальных средствах реализации ПО. \\\hline
\rowcolor[gray]{0.9}
\multicolumn{2}{|l|}{\bfseries \rule{0pt}{0.5cm} 3. ФОРМИРУЕМЫЕ КОМПЕТЕНЦИИ}\\\hline
ПК-2 & способность проводить научные исследования в избранной области экспериментальных и (или) теоретических физических исследований с помощью современной приборной базы (в том числе сложного физического оборудования) и информационных технологий с учетом отечественного и зарубежного опыта \\\hline
\multicolumn{2}{|l|}{Знать:}\\\hline
Уровень 1 & основные термины компьютерного моделирования, предназначение современных математических пакетов \\\hline
Уровень 2 & знать типы данных и их представление, методы работы с физическими данными \\\hline
Уровень 3 & принципы планирования, создания модели и проведения численных экспериментов \\\hline
\multicolumn{2}{|l|}{Уметь:}\\\hline
Уровень 1 & решать типовые математические задачи с использованием математических пакетов \\\hline
Уровень 2 & применять математические пакеты для численного и аналитического решения физических задач \\\hline
Уровень 3 & ставить задачи на разработку программного обеспечения с использованием математических пакетов и решать их \\\hline 
\multicolumn{2}{|l|}{Владеть:}\\\hline
Уровень 1 & навыками работы с основными математическими пакетами для решения стандартных задач \\\hline
Уровень 2 & навыками создания процедур, функций, алгоритмов с использованием математических пакетов для решения физических задач \\\hline
Уровень 3 & навыками разработки программного обеспечения с использованием специализированных пакетов программ, используемых в физике \\\hline
\end{longtable}

\end{document}