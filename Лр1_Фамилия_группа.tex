\documentclass[12pt,a4paper]{article}
\usepackage[utf8]{inputenc}
\usepackage[T1]{fontenc}
\usepackage[russian]{babel}
\usepackage{amsmath}
\usepackage{amsfonts}
\usepackage{amssymb}
\usepackage{graphicx}
\usepackage{epigraph}
\usepackage[left=2.50cm, right=2.00cm, top=2.00cm, bottom=2.00cm]{geometry}

\title{Лабораторная работа $\textnumero 1$\\
	по теме: \LaTeX.}
\author{Студент Иванов А. А., гр. ИСФ(аб)-81}
\date{\today}

\begin{document}
\maketitle

\begin{center}
\large \bfseries
\MakeUppercase{Лев CTAN}\\
\bigskip
\end{center}

\epigraph{Начните с начала, и продолжайте, пока
	не дойдете до конца --- там
	остановитесь\dots}{Льюис Кэрролл, \textit{Алиса
		в стране чудес}}
\bigskip

Эскиз с изображением льва созданный известным коммерческим художником Дуэйном Бибби
использовался в иллюстрациях к книге Дональда Кнута TeXbook, книге Лесли Лампорта \LaTeX,
а также и для других книг, связанных с \TeX.
Лев CTAN хорошо знаком многим в \TeX-сообществе, рисунки с его изображением, красивые и
проницательные, украшали плакаты для конференций, футболки, кофейные кружки.

\begin{center}
\includegraphics[width=0.5\textwidth]{png/ctan_lion_350x350}\\
{\bfseries Рис.1}~Лев CTAN. Автор: Дуэйн Бибби
\end{center}

Вы можете бесплатно использовать этот рисунок. Мы будем признательны, если в качестве благодарности
Дуэйну Бибби вы укажете его авторство при общественном использовании.

\hfill \textit{www.ctan.org}

{\large
\TeX\ -- это компьютерная программа, созданная Дональдом Кнутом (Donald E. Knuth).
}

{\Large
Она предназначена для вёрстки текста и математических формул. Кнут начал писать \TeX\ в
1977 году из-за расстройства от того, что Американское Математическое Сообщество делало с
его статьями в процессе их публикации. Где-то в 1974 году он даже прекратил посылать статьи:
"<Просто мне было слишком больно смотреть на конечный результат">.
}

\TeX, в том виде, в котором мы его используем, был выпущен в 1982 году и слегка улучшен с годами.
Последние несколько лет \TeX\ стал чрезвычайно стабилен. Кнут утверждает, что в нем практически
нет ошибок.

Номер версии \TeX\ сходится к числу $\pi$ и сейчас равен 3.14159265. \TeX\ произносится как "<тех">.\TeX\ -- это компьютерная программа, созданная Дональдом Кнутом (Donald E. Knuth).
Она предназначена для вёрстки текста и математических формул. Кнут начал писать \TeX\ в
1977 году из-за расстройства от того, что Американское Математическое Сообщество делало с
его статьями в процессе их публикации. Где-то в 1974 году он даже прекратил посылать статьи:
"<Просто мне было слишком больно смотреть на конечный результат">.

\TeX, в том виде, в котором мы его используем, был выпущен в 1982 году и слегка улучшен с годами.
Последние несколько лет \TeX\ стал чрезвычайно стабилен. Кнут утверждает, что в нем практически
нет ошибок.

Номер версии \TeX\ сходится к числу $\pi$ и сейчас равен 3.14159265. \TeX\ произносится как "<тех">.

Среднее арифметическое и среднее геометрическое n положительных чисел \(a_1,\ \dots, a_n\) связаны соотношением
\[\sqrt[n]{a_1\cdots a_n}\le \frac{a_1+\cdots+ a_n}{n}\]

\(\int \limits_S \left( \frac{\partial Q}{\partial x} - \frac{\partial P}{\partial y} \right)\, dx \, dy =\oint \limits_C P\,dx + Q \, dy\)

\end{document}