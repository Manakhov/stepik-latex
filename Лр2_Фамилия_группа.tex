\documentclass[12pt,a4paper]{article}
\usepackage[utf8]{inputenc}
\usepackage[T1]{fontenc}
\usepackage[russian]{babel}
\usepackage{amsmath}
\usepackage{amsfonts}
\usepackage{amssymb}
\usepackage{graphicx}
\usepackage[left=2.50cm, right=2.00cm, top=2.00cm, bottom=2.00cm]{geometry}

\title{Лабораторная работа $\textnumero 2$\\
	по теме: Набор простых математических и физических формул в \LaTeX.}
\author{Студент Иванов А. А., гр. ИСФ(аб)-81}
\date{\today}

\begin{document}
	
\selectlanguage{russian}

\maketitle

\section*{Кулоновская волновая функция}

Радиальное уравнение Шредингера для задачи рассеяния электрона с импульсом $p$ на положительном ядре с зарядом $Z$ в атомных единицах имеет следующий вид:
\begin{equation}\label{eq_1}\mathbf{
\left(-\frac{1}{2}\frac{d^2}{dr^2}+
\frac{l(l+1)}{2r^2}-
\frac{Z}{r}-\frac{p^2}{2}\right)\psi=0
}
\end{equation}
Выпишем регулярное решение уравнения \eqref{eq_1}, приведенное в \cite{knut}:
\begin{equation}\label{eq_2}
\varphi_l^c(p,r)=r^{l+1}{e^{ipr}}_1F_1(l+1+it,2l+2,-2ipr),
\end{equation}
где $t=-\frac{Z}{p}$ -- параметр Зоммерфельда для случая взаимного притяжения электронов и ядра. Асимптотическое поведение функции \eqref{eq_2} при $r\to\infty$
\begin{equation}\label{eq_3}
\varphi_l^c(p,r)\xrightarrow[r\to\infty]{}2(2p)^{-l-1}e^{\frac{1}{2}\pi t}
\frac{\Gamma(2l+2)}{\left|\Gamma(l+1+it)\right|}
\sin(pr-t\ln{2pr}-\frac{1}{2}\pi l+\sigma_l)
\end{equation}
где $\sigma_l=\arg\Gamma(l+1+it)$ -- кулоновская фаза.

Радиальная волновая функция, соответствующая асимптотике расходящейся сферической волны, $\psi_l^{c+}(p,r)$
\begin{multline}\label{eq_4}
\psi_l^{c+}(p,r)=\frac{1}{2}(2p)^{l+1}e^{-\frac{1}{2}\pi t}
\frac{\Gamma(l+1+it)}{\Gamma(2l+2)}\varphi_l^c(p,r)=\\=
\frac{1}{2}(2pr)^{l+1}e^{-\frac{1}{2}\pi t}e^{ipr}
{\frac{\Gamma(l+1+it)}{\Gamma(2l+2)}}_1F_1(l+1+it,2l+2,-2ipr)
\end{multline}
связана с трехмерной волной функцией $\psi^{c+}(\vec{p},\vec{r})$ соотношением
\begin{equation}\label{eq_5}
\psi^{c+}(\vec{p},\vec{r})=\sqrt{\frac{2}{\pi p}\frac{1}{r}}
\sum_{lm}^{}i^l\psi_l^{c+}(p,r)Y_lm(\hat{r})Y_lm^*(\hat{p})
\end{equation}
Так как волновая функция \eqref{eq_5} описывает состояния из непрерывного спектра, то состояния с различными значениями импульса $\vec{p}$ ортогональны в том смысле, что выполняется условие нормировки:
\begin{equation}\label{eq_6}
(\psi^{c+}(\vec{p'})|\psi^{c+}(\vec{p}))=\delta(\varepsilon-\varepsilon')
\delta(\hat{\vec{p}}-\hat{\vec{p'}})=p\delta(\vec{p}-\vec{p'}),
\end{equation}
где $\varepsilon=\frac{p^2}{2},\varepsilon'=\frac{p'^2}{2}$ -- энергия электрона.
\begin{equation}\label{eq_7}
0\xleftarrow[\zeta]{\alpha}\mathfrak{F}\times\Delta[n-1]
\xrightarrow[]{\partial_0\alpha(b)}\mathbb{E}^{\partial_ob}
\end{equation}
\begin{equation}\label{eq_8}
M(\varepsilon)=\left|\left|
\begin{matrix}
\varepsilon & 0 & \dotsi & \dotsi & 0 \\
\xi_{n2} & \varepsilon & 0 & \dotsi & 0 \\
\vdots & \vdots & \vdots & & \vdots \\
\xi_{n1} & \xi_{n2} & \dotsi & \xi_{n~n-1} & \varepsilon 
\end{matrix}
\right|\right|
\end{equation}
\begin{equation}\label{eq_9}
\varphi(x)=
\begin{cases}
0  & \text{для}~x \leq0,\\
e^{-1/x} & \text{иначе}.
\end{cases}
\end{equation}
\begin{equation}\label{eq_10}
\overbrace{\underbrace{a+b+\dotsi+z}_{26}+1+\dotsi+10}^{36}
\end{equation}
\begin{thebibliography}{99}
	\bibitem{knut}
	Кнут Д. Е.
	Всё про \TeX.
	\newblock --- Протвино\ : RDTeX, 1993. --- 592~с.
	\bibitem{Stolyarov}
	Столяров А. В.
	Сверстай диплом красиво: \LaTeX\ за 3 дня.
	\newblock --- М.\ : МАКС пресс, 2010. --- 512~с.
\end{thebibliography}

\end{document}